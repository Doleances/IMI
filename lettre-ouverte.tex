\documentclass{report}

\usepackage[utf8]{inputenc}
\usepackage[T1]{fontenc}
\usepackage[francais]{babel}
\usepackage{lmodern}
\usepackage{graphicx}
\usepackage{amsmath}
\usepackage{amssymb}
\usepackage{mathrsfs}
\usepackage{geometry}
\geometry{scale=0.8, nohead}

\author{Nicolas Ehrhardt}
\title{Doléances d'IMI}

\makeatletter
\renewcommand{\thesection}{\@arabic\c@section}
\makeatother

\begin{document}
\maketitle
\chapter{Introduction}

Cela fait maintenant plusieurs semaines que les cours d'IMI se sont terminés. Certains élèves ont déjà commencé leur stage long, d'autres sont encore en vacances, et bien qu'une nouvelle vie débute, les souvenirs de cette dernière année aux ponts restent encore vif.

Durant cette année, il nous est souvent arrivé de débattre sur certains choix concernant notre enseignement et son organisation. À mesure que les idées et les humeurs s'accumulèrent, il nous est apparu opportun de les consigner afin qu'elles puissent être bénéfiques à notre département et à notre école.

Ce document ressence donc idées, avis et humeurs d'élèves d'IMI peu après la fin de leur deuxième année aux Ponts.

\chapter{Description des cours} 

Il existe déjà un service mis en place par monsieur Queixalos dans l'objectif de recevoir l'avis des élèves concernant chaque cours. Et bien souvent il nous est reproché un faible taux de participation. Bien que de nombreux élèves se contentent de remplir le questionnaire succintement ou tout simplement d'ignorer les rappels, aucune alternative n'est proposée. Cette alternative est prise ici, avec toute la liberté que confère la prise individuelle de parole ne subissant pas un effet de lissage par le nombre ou par le politiquement correct.

\section{MOPSI}

Difficile de parler de ce cours sans le diviser en ces trois parties. En outre, étant donné le lien ténu entre les deux parties mathématiques et la partie informatique, il semblerait tout à fait approprié de séparer ces dernières, nous y reviendrons.

\subsection{Analyse (Mr Lelièvre)}
Le cours est très bien documenté de même que le polycopié bien construit. Mr Lelièvre anime les cours avec dynamisme et fait susciter intérêt des élèves pour les mathématiques appliqués. Certains regretteront le manque d'exercice en cours, écho du "vous le ferez en devoir maison" qui est bien souvent oublié ou non fait par manque de temps.

\subsection{Probabilité (Mr Alfonsi)}
Ici aussi, le polycopié est de qualité et monsieur Alfonsi, s'il a moins de charisme que monsieur Lelièvre est tout à fait compétent. Les exercices sont ici un plus gros point noir que dans le cas précédent. En effet, de nouveaux objets avancés de probabilité sont introduits dans ce cours. Malheureusement, les séances ne sont composées presqu'exclusivement de cours, et les exercices corrigés rapidemment au tableau. Si bien que la plupart des élèves en tireront des notions globales mais rares seront capable de manier les martingales ou d'effectuer une intégrale stochastique avec aisance. Ce qui constitue pourtant la majorité du cours.

\subsection{Informatique (Mr Monasse et Mr Marlet)}

Les critiques seront vives. Et c'est sans aucun doute que nous plaidons pour une suppression simple et complète de ce cours. La première raison étant les intervenants. Sur une dizaine d'exercice rendu, aucune correction n'a été faite, et aucune justification de la note ne parvient. Si bien que l'on se demande si un programme de génération de notes aléatoires ne fonctionne pas derrière (ce qui serait tout à fait envisageable au vu de certaines notes: codes similaires, notes non similaires...). Les cours sont assomants, et reprenent des notions parfois déjà vues de manière plus détaillées dans d'autres cours (programmation dynamique en recherche opérationnelle, techniques de développement et optimisation). En outre, le suivi des professeurs est inexistant. De sorte qu'il est bien plus rentable de lire le \verb|siteduzero.com| pendant une heure plutôt que de perdre trois heures en cours.

~

Il pourrait être une très bonne idée de supprimer ce cours, et de nouer un partenariat (ou tout simplement utiliser) le site internet \verb|www.siteduzero.com|. Ce site proposant à notre sens des tutoriels de très bonne qualité en tous langages (C++ inclus) il est aisé de fournir une version papier à chaque élève. Ainsi, un élève voulant améliorer ses compétences en programmation se voit munit d'un bel outil et proposé le cours de techniques de développement logiciel, et un élève plutôt réticent pourra se contenter du strict minimum proposé par les tutoriels pour survivre dans les matières informatiques.

\section{Recherche opérationnelle}

\section{Techniques de développement logiciel}

Bon cours. La première séance de cours est souvent assomante, mais avec le recul très enrichissante pour qui prend le temps d'écouter.

\section{Analyse spectrale}

\section{Maillages et application}

\section{Apprentissage automatique}



\end{document}

